\documentclass{article}
\usepackage{oconnor}
\usepackage{ wasysym }

%% UPDATE these variables:
\renewcommand{\hwnum}{1}
\title{CSCI 338, Exercise 01}
\author{Patrick O'Connor}
\collab{n/a}
%%\date{due: 15 January 2021}

\begin{document}

\maketitle

CSCI 338 Computer Science Theory
Questions on Proof Methods (45 minutes)

% ============================================
% ============================================
\nextprob{Question 1}
\collab{n/a}
% ============================================
Given a simple graph $G = (V,E)$, if $|V| = n$ and $|E| \geq n$ then G must contain a cycle.
\paragraph{Answer}
Restricting this to a simple graph disallows looping edges, multi-edges 
and that the graph is undirected - then the property holds.

Consider a complete graph $K_n$ (with n vertices): each of the n vertices is incident to the other
 $n-1$ vertices via a connecting edge therefore we $n(n-1)$ connections from one vertex to 
 another; given that edges are undirected then this will count each edge twice (i.e counting 
 from vertex A to vertex B and vice versa) then the total number of edges will be $(n(n-1))/2$.

Any graph $G_n$ (with n vertices) will be a sub-graph of $K_n$ (since you cannot add any more 
edges to $K_n$ without creating multi- or looping edges) then there must be equal or fewer 
edges in $G_n$ than in $K_n$.

Thus $E \leq n(n-1)/2$ for all simple graphs.

% ============================================
% ============================================
\nextprob{Question 2}
\collab{n/a}
% ============================================
Peter makes a claim “If I have a ball absolutely round in my hand, then within 30
seconds I can raise the temperature in Bozeman by 20 degrees.” How do you proceed to
find a counterexample for this claim?
\paragraph{Answer}
A counterexample for this would be 
\begin{enumerate}
    \item If an absolutely round ball is held, the temperature within 30 seconds the temperature in Bozeman with increase by 20 degrees
    \item Bozemans temperature increased by 20 degrees due to season change
    \item An absolutely round ball must have been held to change the season.
\end{enumerate}


% ============================================
% ============================================
\nextprob{Question 3}
\collab{n/a}
% ============================================
% ============================================
Prove that $1^(3) + 2^(3) + 3^(3) + \ldots + n^(3) = 1/4n^(2)(n + 1)^(2)$
\paragraph{Answer}

Base case: let $n=1$ then 
$RHS= 1^3 \ldots 1^3$
$=1$

$LHS= (1(1 + 1)/2)^2$
$=1$

So for $n=1$ this holds true as $1=1$

Induction hypothesis:
Assume the formula holds true for up to n = k. That is, assume that
$1^3 + 2^3 + 3^3 + ... + k^3 = [k(k + 1)/2]^2$

In order for induction to prove this we need to show that the formula holds 
true for $n=(k+1)$ 




\end{document}

