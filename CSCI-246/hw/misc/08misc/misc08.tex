\documentclass{article}
\usepackage{../../fasy-hw}
\usepackage{ wasysym }

%% UPDATE these variables:
\renewcommand{\hwnum}{0}
\title{Discrete Structures, Misc 08}
\author{Patrick O'Connor}
\collab{n/a}
\date{due: 26 April 2021}

\begin{document}

\maketitle

This homework assignment should be
submitted as a single PDF file both to D2L and to Gradescope.

General homework expectations:
\begin{itemize}
    \item Homework should be typeset using LaTex.
    \item You will not plagiarize.
    \item List collaborators at the start of each question using the
        \texttt{collab} command.
    \item What did you learn?
    \item     What is a piece of advise that you took away?
    \item     Did an audience member ask a question that you particularly liked or disliked?
    \item    Is there a question that you wish you could have asked but didn't?
    \item     If related to the event, what were you thinking about as you were leaving?
    \item   Did this event inspire you to look something up after the event? If so, what was it?
    \item    Did you meet someone new at this event who you could see as a mentor / mentee / collaborator in the future?
    \item     What is your take-away message from this event?
    \item    How did the contents of this event relate to our class? (Even if not explicitly discussed).
\end{itemize}

% ============================================
% ============================================
\nextprob{Botany of Math}
\collab{n/a}

Visit the art exhibit The Botany of Math at the Emerson. Write a 1-2 page reflection on your experience.

On Thursday of this past week I visited The Botany of Math at the Emerson. I had originally tried to go the previous weekend 
but was disappointed to find out it was only open on the weekdays after 10am. Even with this mistake, I was excited to 
return after seeing the inside through the glass doors. 

When I was finally able to go through the doors, I enjoyed seeing the art that was displayed throughout the medium sized room.
Some of the pictures I took can be seen below. Although I enjoyed taking some time off from school and looking at some art. 
I found that the exhibit lacked substantial connection between nature/botany and mathematics. I was extremely disappointed
in the musuem and felt that many of these pieces were simply a mathematical concept that had been displayed with some
leaves instead of simple coloring and pencil work. 

Although as a whole I did not see the connection in most of pieces I found the shell comparision with the fibanocci sequence 
to be one of the better pieces. I enjoyed looking at the variety of shells presented nicely with a neat graph behind it.
Even with this being my favorite, I found that this is one of the most common examples of 
where mathematics can be seen in nature and therefore is an elementary example that is overused. 

Overall, I was upset with the quality of this exibit and wish that the artist creating it would have reached out to a 
professional in a mathematics field to try and gain knowledge on more complex examples of where Math is found in Botany.
Although this would have been changing the artist's full creative process I dont think that this exibit was complete and needed
some professional consultation.  


\begin{figure}\centering
    \subfloat[Fibanocci Sequences in Shells]{\label{fib}\includegraphics[width=.45\linewidth]{fib.png}}\hfill
    \subfloat[Labyrinthitis, Hearing Loss, and Greek Mythology]{\label{lab}\includegraphics[width=.45\linewidth]{spiral.png}}\par 
    \subfloat[Turans Brick Fractory Problem]{\label{turan}\includegraphics[width=.45\linewidth]{brick.png}}
\end{figure}

% ============================================
% ============================================



\end{document}

