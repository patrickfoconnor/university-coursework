\documentclass{article}
\usepackage{../../fasy-hw}
\usepackage{ wasysym }

%% UPDATE these variables:
\renewcommand{\hwnum}{G}
\title{Discrete Structures, Misc G}
\author{Patrick O'Connor}
\collab{n/a}
\date{due: 15 January 2021}

\begin{document}

\maketitle

This homework assignment should be
submitted as a single PDF file both to D2L and to Gradescope.

General homework expectations:
\begin{itemize}
    \item Homework should be typeset using LaTex.
    \item You will not plagiarize.
    \item List collaborators at the start of each question using the
        \texttt{collab} command.
    \item What did you learn?
    \item     What is a piece of advise that you took away?
    \item     Did an audience member ask a question that you particularly liked or disliked?
    \item    Is there a question that you wish you could have asked but didn't?
    \item     If related to the event, what were you thinking about as you were leaving?
    \item   Did this event inspire you to look something up after the event? If so, what was it?
    \item    Did you meet someone new at this event who you could see as a mentor / mentee / collaborator in the future?
    \item     What is your take-away message from this event?
    \item    How did the contents of this event relate to our class? (Even if not explicitly discussed).
\end{itemize}

% ============================================
% ============================================
\nextprob{Anna's Practice Talk}
\collab{n/a}
% ============================================
% ============================================

Anna Schenfisch a mathematics graduate student who has assisted Professor Fasy by subbing in during one of our lectures gave her PhD practice talk on Tuesday, 6 April 
on the Posets of Topological Descriptors. This talk felt way above my knowledge of mathematics and topology but I did enjoy attending it as it was a chance to learn 
more advanced mathematics and gave me a couple things to look into after attending.

I enjoyed attempting to understand what Anna was exactly talking about even though there were alot of terms that I had never had any education on. As an uneducated viewer 
I found that the topological descriptor types was quite similar to the irreducible sets that were used in proving the four color thereom proof that we read about in 
our class book Four Color Thereom.

As someone who has never experienced a PhD seminar I enjoyed seeing the work that is surrounding these talks in the background. From simple changes like choosing a better 
color or even noticing that just because someone is an extremely educated does not mean that they cannot make mistakes. Anna has done immense amount of research into the 
background of this problem yet she had used an incorrect definition for Omega. This gave me faith that although I do make mistakes in many of our assignments in this class 
it is ok and there is a lot of jargon and definitions that come along with this specific field. From this I have been able to reinforce in my mind that this is a complex 
field and it will take alot of work to fully have concrete definitions in my toolbox and it is ok to not understand and reach for help from those that know than me.

Lastly, I found some different research topics that I hope to dive deeper into during our summer break and my next discrete mathematics class that I will be taking over 
that break. Before this class and Anna's talk I had never heard any information on topological descriptors and find it to be an interesting example of how these mathematical 
representations can be used to study and discover ideas about the nature and our surroundings.

\end{document}

