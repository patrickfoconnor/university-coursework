\documentclass{article}
\usepackage{fasy-hw}

%% UPDATE these variables:
\renewcommand{\hwnum}{4}
\title{Discrete Structures, Homework \hwnum}
\author{{Patrick O'Connor} (Patrick OConnor (322))}
\collab{n/a}
\date{due: 5 March 2021}

\begin{document}

\maketitle

This homework assignment should be
submitted as a single PDF file both to D2L and to Gradescope.

General homework expectations:
\begin{itemize}
    \item Homework should be typeset using LaTex.
    \item Answers should be in complete sentences and proofread.
    \item You will not plagiarize.
    \item List collaborators at the start of each question using the \texttt{collab} command.
    \item Put your answers where the \texttt{todo} command currently is (and
        remove the \texttt{todo}, but not the word \texttt{Answer}).
\end{itemize}


% ============================================
% ============================================
\collab{n/a} \nextprob{Good Proofs}
% ============================================
% ============================================

Look through proofs in this textbook, or other books / papers.  Define five
qualities that you think are common among good proofs. Provide citations to
examples.


\paragraph{Answer}
\begin{enumerate}


  \item Secondly including more than just the logic that is required for the proof.\
      As logic only proves that an argument is valid and not if it sound
        \emph{Textbook1}~\cite{etextbook1}

  \item Do not be afraid to prove something in multiple ways.
    \emph{Textbook2}~\cite{etextbook2}

  \item The first quality that is common in good proofs is to have a clear starting point
    or end point or both if possible.
    \emph{Verbal}~\cite{fasylecture}

  \item Write in paragraph form
  \emph{Textbook3}~\cite{etextbook3}

  \item Displaying equations and inequalities increase the readability of a proof
    \emph{Textbook4}~\cite{etextbook4}
\end{enumerate}

{Five Qualities that are common among good proofs: }


{Clarity,good proofs communicate their ideas with as much detail as necessary.}

{Logical, the logic behind the proof is sound and mathematically or logically correct.}

{Professional, correct notations, punctuation, and grammer is used.}

{Breif, good proofs use consise explanations.}

{Significant, the proof reveals important ideas related to the topic.}


{Citation to explamples: "Discrete Mathematics: An Open Introduction by Oscar Levin section 3.2" 2021)


% ============================================
% ============================================
\collab{n/a} \nextprob{Max of a Subset}
% ============================================
% ============================================

Let $(B,\leq)$ be a totally ordered finite set. Prove the following
statement: For all nonempty subsets $A \subseteq B$, the following inequality
holds: $\max(A) \leq \max(B)$.

\paragraph{Answer}

Let $a$ be the $\max \in$ finite set $A$ and $b$ be the $\max \in$ finite set $B$
If A is finite, then $\max$ A $\in$ A by definition of maximum of a finite set.
If B is finite, then $\max$ B $\in$ B by definition of maximum of a finite set.

By definition of subsets element $A \in B$. Therefore $a \in B$ but b is not always
an $\in A$ because set A's range is less than set B's range by given prompt.

Hence b $>$ a and $\max(A) \leq \max(B)$.

% ============================================
% ============================================
\collab{Dastan Riazati} \nextprob{Fibonacci}
% ============================================
% ============================================

The Fibonacci numbers are defined as follows:
$$
    F_i = \begin{cases}
            1 & i \in \{1,2\} \\
            F_{i-1}+F_{i-2} & \text{otherwise}
          \end{cases}
$$

Prove $\sum_{i=1}^n F_i = F_{n+2}-1$.

\paragraph{Answer}

With a true base case of n = 1.

Left hand: $\sum_{i=1}^1 F_1 = 1$
Right hand: $F{3}-1 = F{2}+F{1}-1 = 1+1-1 = 1$
Therefore this true for n=1

Assuming $\sum_{i=1}^k Fi = F{k+2}-1$

Now, prove $\sum_{i=1}^{k+1} Fi = F{k+3}-1$

$\sum_{i=1}^{k+1} Fi = (\sum{i=1}^k Fi) + F{k+1}$ by definition of summation.

$(\sum_{i=1}^k Fi) + F{k+1} = F{k+2}-1 + F{k+1}$ by induction hypothesis.

$F{k+2}-1 + F{k+1} = F{k+3}-1$ by definition of Fibonacci numbers.

Thus, $\sum{i=1}^{k+1} Fi = F{k+3}-1$

Thus, $\sum_{i=1}^n Fi = F{n+2}-1$ by induction.

% ============================================
% ============================================
\collab{n/a} \nextprob{US Coins}
% ============================================
% ============================================

Consider the four smallest denominations of US coins: $D=\{1,5,10,25\}$.  Prove, using
induction, that, for each $n \geq 1$, you can make $n$ cents using at most four
pennies.

\paragraph{Answer}

Given the four smallest denominations of US coins: $D=\{1,5,10,25\}$ We can create
a formula in relation to n as $n = a+5b+10c+25d$ where each coin is represented respectively
in the formula followed by a letter for how many.

Base case: $P(1) = 1(1)+5(0)+10(0)+25(0) = 1$

Assume that P(n) is true in that $n = a+5b+10c+25d$.

With that assumption we can move to n+1 which can be represented as
$P(n+1) = a_1+5b_1+10c_1+25d_1$ where $a_1=a+1$.

Moving towards $n=k$, we can assume that $k = a^1+5b^1+10c^1+25d^1$ is true.

Proving that $n = k+1$ and $n =k+9$
For $n = k+1$ we can use the previous statement of $k = a^1+5b^1+10c^1+25d^1 = k+1$

For $n = k+9$ we will need $a^1=4$ and $b^1=1$

To prove our statement of we need at most four pennies to make $n$ cents we can
use the $n = k+9$ and remove the $b^1=1$ so that $b^1=0$. Expressed fully as
$n = 1(1)+5(0)+10(0)+25(0)$

In conclusion, to make any cents $n$ we require at most four pennies along with
an assortment of other denominations.
By proof by mathematical induction P(n) is true.


% ============================================
% ============================================
\collab{Dastan Riazati} \nextprob{Four Colors Suffice}
% ============================================
% ============================================

Read Chapters $4$ and $5$ of \emph{Four Colors Suffice}.

Use a proof by contradiction to prove that if an edge is removed from a
tree, then the resulting graph has two connected components.

\paragraph{Answer}
By definition a tree must be acyclic and the number edges can be found by v-1
where v is the number of vertices.

Our statement above is false at n edges and n+1 vertices, minimum such. But at
n-1 edges, n vertices our statement is true in that if the edge is removed two components will
be found in the resulting graph. This resulting graph after removing can be
expressed as having a total of v-2 edges and still v vertices. By previously
stated definition, in order for a tree to remain a tree after manipulation it must
be acyclic and the number of edges must be v-1. Therefore our previous tree no
longer fits the definition of a tree unless the original tree is split into two
connected components. Along with this when splitting a tree at say previously
connected vertices $v_1$ and $v_2$ since by definition there is no cycles, these
two vertices are no longer connected in any way by an edge or it would have not
been a tree to begin with.

Therefore we can conclude that if there is a tree G that has an edge taken away
the resulting graph will be two connected components.

{A tree is an acyclic graph which means it has no cycles and it is connected. Following this definition a tree has to be a simple graph.}

{We know a graph is a tree if there is absoluty 1 path between all pairs of the vertices}

{Let n = a graph such that there is exactly 1 path between all pairs.}

{Therefore n is connected, meaning no cycles}


{Let a and b be verticies of n}

{Between a and b there are 2 paths which is the contridiction}

{The path between this pair contains a cycle which contradicts one of our proproties of a tree (A tree is an acyclic graph).


% ============================================
% ============================================
\collab{n/a}
\nextprob{Leonhard Euler}
% ============================================
% ============================================

Write a short (1-2 paragraph) biography of Leonhard Euler.
\textbf{In your own words}, describe who they are and why they are important in
the history of computer science.

If you use external resources, please provide
proper citations. If you do not use external sources, please write ``I did not
use any sources to write this biography'' as the last sentence of the
biography.

\paragraph{Answer}

Leonhard Euler was extremely diverse learner and expert, classified as a
mathematician, physicist, astronomer, geographer, logician, and engineer.
Euler started his life in Switzerland in 1707 where he originally started his
professional career as a rural clergyman. With proof of great aptitude in
mathematics. Johan Bernoulli, of Bernoulli principle, recognized this greatness
in the making and pulled him to study under him. After studying under Bernoulli,
Euler attended University of Basel in his home town. After earning his masters
in his teens, Euler moved to Russia and served in the navy. After serving in the
navy he returned to the educational system in order to spread his knowledge as a
professor in physics and mathematics.

During this period, Euler made expansive contributions to mathematics(geometry,
calculus, trigonometry, algebra, and number theory) and physics. Euler was the
first to introduce the notation of f(x) and popularized $\pi$ to denote the ratio
of a circle's circumference to its diameter. Euler's contribution to graph theory
and topography is a large contribution to the current systems that are in place
for route finding applications and any mapping application. Euler made these
discovery through investigating Königsberg Bridge Problem and Knight's Tour Problem.
The Königsberg Bridge Problem investigated whether it was possible to walk around
the entirety of Königsberg using all of the seven bridges using each bridge exactly
once. Euler concluded that this was in fact impossible and displayed this by picturing
the landmasses as nodes and the bridges as edges. Further Euler proved that if
the land masses/nodes were an even number with an odd number of bridges this
would be always possible. The Knights Tour Problem is similar
in that it can be represented as a graph with nodes and edges but rather the
intention to move the knight to each space exactly once using the class knight
move. This initial graph theory creation is found in many complex fields of computer
science currently such as telephone switching systems and the investigation of
the Four Color Theory.
\emph{Encyclopedia}~\cite{encyclopedia}
\emph{Britannica}~\cite{britannica}


% %% ... the bibliography
 \newpage
 \bibliographystyle{acm}
 \bibliography{biblio}

\end{document}
