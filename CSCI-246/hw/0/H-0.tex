\documentclass{article}
\usepackage{fasy-hw}
\usepackage{ wasysym }

%% UPDATE these variables:
\renewcommand{\hwnum}{0}
\title{Discrete Structures, Homework 0}
\author{Patrick O'Connor}
\collab{n/a}
\date{due: 15 January 2021}

\begin{document}

\maketitle

This homework assignment should be
submitted as a single PDF file both to D2L and to Gradescope.

General homework expectations:
\begin{itemize}
    \item Homework should be typeset using LaTex.
    \item Answers should be in complete sentences and proofread.
    \item You will not plagiarize.
    \item List collaborators at the start of each question using the
        \texttt{collab} command.
\end{itemize}

% ============================================
% ============================================
\nextprob{Getting to Know You}
\collab{n/a}
% ============================================
% ============================================

Answer the following questions:
\begin{enumerate}
    \item What is your elevator pitch?  Describe yourself in 1-2
        sentences.
        \paragraph{Answer} My name is Patrick O'Connor and I am computer science major with a minor in sustainability. I have always loved the outdoors(fishing, boating/kayaking/canoeing, hiking, and more). Because of this I have decided that I will go into the field of Computational Sustainability and hope to help preserve the environment for the next generations. 

    \item What was your favorite college class so far, and why?
        \paragraph{Answer} My favorite college class so far has been CSCI 232. I found it to be extremely interesting and very different than any other class I have ever taken. I found the thought process behind algorithms to be very thought provoking and could feel my thought process of how I write code changing as I found out more about algorithms and the creation of data structures. 

    \item What was your least favorite college class so far, and why?
        \paragraph{Answer} Environmental Biology has been my least favorite college class. I find straight memorization to be boring and would rather learn about systems and how each part fits into the larger picture compared to memorizing figures.

    \item Why are you interested in taking this course? (If your answer is
        `because I am required to by my major/minor', perhaps answer the
        alternative question: Why are you in your major?)
        \paragraph{Answer} I am interested in this course as I see it as a way for me to increase my problem solving skills and expand my options of working through problems.

    \item What is your biggest academic or research goal for this semester (can
        be related to this course or not)?
        \paragraph{Answer} My largest academic goal for this semester is to increase my computer science tool box and problem solving skills. Along with this as a general goal I plan to achieve above a 3.75 gpa.

    \item What do you want to do after you graduate?
        \paragraph{Answer} I am not exactly sure if I want to be in the private or education/research sector but I want to work in computational sustainability and hope to assist in creating software that can solve major environmental issues such as energy, agriculture, and land use. I plan to enter the private sector most likely and then go back to school for a master/PhD. After this I hope to be well off and then go into educating and doing research through a university. 

    \item What was the most challenging aspect of blended or online courses?
        \paragraph{Answer} The most challenging aspect of blended/online is following along with all of the different platforms that teachers have decided are the best fit for their class. Thank you for asking for input from your students and deciding to use discord. It is greatly appreciated.

    \item What do you like about blended or online courses?
        \paragraph{Answer} I enjoy having the oppurtunity to use my computer setup that I have at home. Being able to use this allows for a much cleaner and efficient workflow.

\end{enumerate}

% ============================================
% ============================================
\nextprob{Administrative Tasks}
\collab{n/a}
% ============================================
% ============================================

Please do the following:
\begin{enumerate}
    \item Write this homework in LaTex. This will not be strictly enforced for
        this homework, but it is strongly encouraged.  Future homeworks will not
        be graded if they are not typeset in LaTex.
    \item Update your photo on D2L to be a recognizable headshot of you.
    \item Sign up for the class discussion board.
\end{enumerate}

\paragraph{Answer}


I have completed this homework assignment using LaTex, vim, and the MakeFile provided.

% ============================================
% ============================================
\nextprob{Plagiarism}
\collab{n/a}
% ============================================
% ============================================


In this class, please properly cite all resources that you use.  To refresh your
memory on what plagiarism is, please complete the plagiarism tutorial found
here: \url{http://www.lib.usm.edu/plagiarism_tutorial}.  If you have observed
plagiarism or cheating in a classroom (either as an instructor or as a student),
explain the situation and how it made you feel.  If you have not experienced
plagiarism or cheating or if you would prefer not to reflect on a personal
experience, find a news article about plagiarism or cheating and explain how you
would feel if you were one of the people involved.

\paragraph{Answer} I personally have only experience feedback on citing sources properly. I found this feedback to be extremely helpful and have learned a good amount about citing sources correctly.

% ============================================
% ============================================
\nextprob{Exams}
\collab{n/a}
% ============================================
% ============================================

I am exploring various options for exams for this semester: take-home,
in-person, synchronous online.  If you have any comments about what worked or
did not work in previous semesters with respect to classes in blended and online
settings, please share that here.

\paragraph{Answer}


I have found that if the class meets online the majority of the time it can be challenging to complete in-class quizzes & exams. In my personal experience it is the change of environment that causes me the most challenges. If I do everything sitting at a desk in class I want the test to be in that classroom. While if I do everything at home I prefer to be in the same chair when taking the quiz or exam. I do not have a preference for when the exam is (During our class-time or open for a day) although technology problems have happened in the past and it cause quite alot of stress as the teacher did not have any sort of leeway in the test time. Ex. 50 questions in 50 minutes. If the computer works the way it should but anyone that works with computers on a daily basis knows this is not always the case.




% ============================================
% ============================================
\nextprob{Terminology}
\collab{\todo{}}
% ============================================
% ============================================

Sometimes concepts are taught more than once throughout the curriculum.  Each
time you encounter a concept, your understanding of it is deepened.
For each of the terms or statements below, describe in your own words what they
mean.  This will not be graded for correctness, just whether you have done it or
not.  Answering these to the best of your ability will help the instructor and
TA understand the base knowledge of the students in this class.
I encourage you to meet with a partner or two to refresh yourself on what these
terms mean (if you do, be sure to update the \texttt{collab} command
above!).  However, please keep the web searches to a minimum for this one!  It
is acceptable to answer `I have not heard of this term' or `I have heard of
this, but do not remember what it means.'
\begin{enumerate}
    \item $f(n)$ is $O(n^2)$.
        \paragraph{Answer}
        f(n) is big O of (n^2). I understand what this means but do not have a concrete enough of an understanding to define it.
    \item $f(n)$ is $O(g(n))$.
        \paragraph{Answer}
        f(n) is big O of g(n). I understand what this mean but do not have a concrete enough of an understanding to define it.
    \item $f(n)$ is $\Omega(n^3)$.
        \paragraph{Answer}
        I have not heard of this term
    \item $f(n)$ is $\Theta(n\log n)$.
        \paragraph{Answer}
        I have not heard of this term
    \item Binomial Coefficients
        \paragraph{Answer}
        I have heard of this, but do not remember what it means.
    \item Four Color Theorem
        \paragraph{Answer}
        I have only heard of this term after looking into the required text.
    \item Graph
        \paragraph{Answer}
        A set of ordered pairs.Example in regards to 2D planes (x, f(x)). 
    \item Modus Ponens
        \paragraph{Answer}
        I have not heard of this term.
    \item Proof by Counter-example
        \paragraph{Answer}
        Simply put it is using cases that disprove a statement to prove that it there are contradictions. It has been awhile since I have done any proofs.
    \item Proof by Example
        \paragraph{Answer}
        Bad way to prove something. Providing a case or multiple cases to prove that a statement is True.
    \item Proof by Induction
        \paragraph{Answer}
	I have heard of this term but need to do a refresher.        
    \item Recurrence Relation
        \paragraph{Answer}
        I have not heard of this term.
    \item Recursive Algorithm
        \paragraph{Answer}
        An algorithm that calls itself to break down the problem into smaller and simpler parts. Resulting in a solution once getting base case.
    \item Searching Algorithms
        \paragraph{Answer}
        Algorithm used to search and retrieve information
    \item Sorting Algorithms
        \paragraph{Answer}
        Algorithm used rearrange a given set of elements based on a comparision operator.
    \item Tree
        \paragraph{Answer}
        A collection of nodes place in a non-linear data structure.
\end{enumerate}

% ============================================
% ============================================
\nextprob{Real Numbers}
\collab{N/A}
% ============================================
% ============================================

Review the Properties of Real Numbers in Appendix A.  If any are unfamiliar or
confusing, please post a question in the group discussion board.  In the
write-up, write the following: `I have reviewed all properties of real numbers
in Appendix A.`

\paragraph{Answer}

I have reviewed all properties of real numbers in Appendix A.

% ============================================
% ============================================
\nextprob{Georg Cantor}
\collab{N/A}
% ============================================
% ============================================

Write a short (1-2 paragraph) biography of Georg Cantor.
\textbf{In your own words}, describe who they are and why they are important in
the history of computer science.  If you use external resources, please provide
proper citations.

\paragraph{Answer}

% ============================================

George Cantor is a Russian born mathematician that founded set theory and was the introducer of transfinite numbers. He was born in Russia in St. Petersburg in March, 1845. He was raised by a musically inclined mother and sales focused father. By the age of 15, Cantor's mathematical skills began to flourish and were recognized while studying in a private school. Around the same time his family decided that it was time to move to Germany after their father began to get ill.

After rejecting his fathers request to become an engineer, Cantor transferred to the University of Berlin to specialize in physics, philosphy, and mathematics. With a specialization of mathematics under his belt, he went back to school and shortly after attending started a doctoral thesis on the indeterminate equations of the second degree. Once completed, he went on to write a series of 10 papers from 1869 to 1873. These 10 papers included but were not limited to the theory of trigonometric series, set theory, and transfinite numbers. His two best known works are the expansion of set theory and the creation of tranfinite numbers. With a commitment to expanding mathematics, Cantor remained actively at work fighting to ensure future generations would not have to suffer as he did with faculty members that were threatened by new ideas. Overall he has been a massive stimulation point and has altered the education of mathematics.
 
Bibliography: O'Connor, JJ, and E F Robertson. “Georg Cantor - Biography.” Maths History, Oct. 1998, mathshistory.st-andrews.ac.uk/Biographies/Cantor/Bibliography. 

% ============================================

\end{document}

