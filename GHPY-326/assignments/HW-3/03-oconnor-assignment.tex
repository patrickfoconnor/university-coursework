\documentclass{article}
\usepackage{oconnor-hw}

%% UPDATE these variables:
\renewcommand{\hwnum}{3}
\title{GPHY 326 Geography of Energy Resources, Energy Poverty in
Montana, Homework  \hwnum}
\author{{Patrick O'Connor}}
\collab{n/a}
\date{due: Friday 3/5/2021, 5pm}

\begin{document}

\maketitle

The purpose of this assignment is to help you explore and apply concepts about
energy poverty to information about the US and the state of MT. By using
recommended resources to answer a set of questions about energy poverty in the
US and our state, you will learn: how experts understand energy poverty in the
US, patterns of energy poverty in the US and Montana, and the policies available
to address the direct impacts of energy poverty at the state and national level.

Suggested process: :
\begin{itemize}
    \item download the worksheet--this is the file to use to submit your
    homework
    \item access and/or download the two starting sources, shown above
    \item scan each to gain a familiarity of the kind of information covered
    in the document/web page
    \item read the list of questions and make a plan for how you will answer
    them
    \item Implement your plan. Be sure to read carefully and thoughtfully. 
\end{itemize}

For future reference:
Understanding energy poverty in the US:
\begin{itemize}
  \item The US Dept of Energy's \href{https://www.energy.gov/eere/slsc/maps/lead-tool}{Low-Income
  Energy Affordability Data (LEAD) Tool}.
  \item The American Council for an Energy-Efficient Economy (ACEEE)'s
  \href{https://www.aceee.org/sites/default/files/pdfs/u2006.pdf}{2020 US
  Household Energy Burden Report}.
  \item \href{https://headwaterseconomics.org/tools/populations-at-risk/}
  {Headwaters Economics' Populations at Risk data tool }.
\end{itemize}
Federal energy assistance programs:
\begin{itemize}
  \item LIHEAP \href{https://dphhs.mt.gov/hcsd/energyassistance}{information for
  Montana}.

\end{itemize}


% ============================================
% ============================================
\collab{n/a} \nextprob{Household Energy Burden}
% ============================================
% ============================================

1) What is the standard definition of household energy burden used by the DOE
and ACEEE? What data are used to calculate household energy burdens in the two
data sets—and what are the sources of these data?

\paragraph{Answer}

The DOE and ACEEE define household energy burden as the share of annual
household income that is used to pay annual energy bills. This can be further
detailed depending on the percent of income that it used. Energy burden is
considered high if the percentage used to pay energy bills is greater than
6\% of income and severe if greater than 10\%.


The data that are used to calculate household energy burden inhttps://www.aceee.org/sites/default/files/pdfs/u2006.pdf the two data sets
are the biennial AHS and the U.S. Census 5-year American Community Survey (ACS5)
. The AHS is a household-level survey that collects data on household-level
income and energy cost data that are utilized by the ACEEE as a base-point
for calculating their energy burden statistics. The AHS recently in 2015 stopped
collecting self-reported energy cost as part of their survey and rather use
estimations energy costs through regression-model-based imputation. This utility
estimation system is based on administrative data collected directly from the
suppliers actual yearly billing amounts. With a simple arithmetic operation this
yearly stat can be turned into an average monthly cost. The U.S. Census 5-year
ACS5 is utilized by DOE to create data sets including the LEAD tool. The DOE estimates residential
energy consumption by cross-examining/cross-tabulating Census housing data. With
this data they can apply a weight to the data based on the the housing unit type
and the housing counts. With this the DOE can create census tract-level
estimates of energy expenditures that we will examine in problem 3 \& 4. The
cross-tabulation and cleaning the data before running the cross-tabulating are
key parts to the DOE getting accurate results from an extremely complex system.


% ============================================
% ============================================
\collab{n/a} \nextprob{Household energy burden and energy poverty}
% ============================================
% ============================================

How does the concept of household energy burden relate to definition of energy
poverty used in the textbook?

\paragraph{Answer}

The concept of household energy burden is closely related, to the point
of running almost parallel to the definition of energy poverty used in Energy and
Society. The book initially used a definition that was almost a word for word
copy in that if the household needs to spend more than 10\% of their total
household income before housing cost they are within energy poverty. The updated
definition in the book is reconstructed to go past examining a percentage and
rather looks through the lens of if they have above-average fuel costs which,
if spent fully, would leave them with a residual income below the official
poverty line.

% ============================================
% ============================================
\collab{n/a} \nextprob{Using the DOE LEAD tool}
% ============================================
% ============================================

According to the DOE LEAD tool, in which counties are household energy burdens
highest in Montana? Does this geography change at the scale of census blocks?

\paragraph{Answer}

According to the DOE LEAD tool, Meagher, Treasure, and Carter county have the
highest energy burden in Montana at \%9. When using the census blocks this data
does specify different locations than before with the majority of them still
being in eastern Montana. The following census regions have greater values then
previously found in the counties map.
\begin{itemize}
    \item  \%10: Census Tract 9407 in Big Horn County \& Census Tract 9404 in Rosebud County
    \item  \%11: Census Tract 9403 in Hill County
    \item  \%14: Census Tract 9402 in Blaine County
\end{itemize}


% ============================================
% ============================================
\collab{n/a} \nextprob{DOE LEAD tool Exploration}
% ============================================
% ============================================

Select a county or a census block of interest to you from the DOE LEAD tool.
Now, use the data provided by the “Populations at Risk” tool to explore and
understand the local context.


Based on the data available in the Populations at Risk tool, which do you think
might be important local social and economic variables that influence household
energy burdens for the location you chose?

\paragraph{Answer}

I chose Blaine County as I was intrigued by the fact that although it holds
the highest energy burden percentage region when looking at census tract it sits
below the definition of a state of energy burden for occupants at /%6 when
examining at a county level.

Utilizing the Headwaters Economic: Populations at Risk data curation the
largest local social and economic variables that influence Blaine County are
the households demographics. When examining Blaine County I discovered that
the majority of residents were of the minority group. This is not specifically
a reason to assume that energy burden is present but worthy in considering the
following points. Among the residents that called Blaine home there is a
disproportionate amount of poverty and deep poverty compared to the U.S..
As of 2019 Blaine currently has \%30.5 of residents living in poverty while the
U.S. is roughly at \%13.4 impoverished. Along with this the county has roughly
three times the national average of occupants living in deep poverty(Blaine:
\%15.9, U.S. \%6). This imbalance of poverty is directly related to compromised
health and the ability to recover from instabilities. Along with this a closely
related statistic found is that roughly people 19.9\% of residents live
with disabilities compared to the U.S. average of 12.6\%. This is an essential
component in considering energy burden on a community as although not all
disabled people have problems with finding stable jobs there are some disabilities
that inhibit a household member from working full-time or even part-time. If a
member of the household is not available to work yet still using energy a larger
percent of those actively working household member's salary will be used to maintain
live-able conditions. Lastly, although our definition excludes housing cost as
they used pre-expensed salary, housing cost is extremely high in the area compared
to the annual salary. On average in roughly 38\% of Blaine County households
spend greater than $>30\%$ of household income on Mortgage alone. Compared to the
U.S. average of 27.7\%.

With an expansive proportion of residents living in poverty for a variety of
reasons providing an affordable energy is extremely hard. With a limited amount of income due to
disabilities and expenses increasing in health-care and housing. The community
of Blaine is currently stuck in a cycle of being impoverished and therefore
will continue to have an energy burden until changes are made within the
community and local, state, and federal legislation to help those in need. The
LIHEAP is currently working to decrease energy burden
According to Montana.gov there is an LIEAP office in Havre, Montana which is
near Blaine County.


% ============================================
% ============================================
\collab{n/a} \nextprob{LIHEAP and energy burden reduction programs}
% ============================================
% ============================================

Using what you can discover about household energy assistance programs in
Montana, and after reading the ACEEE report, please discuss the following:

What is LIHEAP and to what extent does it address the energy burden concerns
described in the ACEEE assessment of national household energy burdens? To what
extent do energy assistance programs in Montana appear to target critical issues
of energy poverty in our state?

(Three or more paragraphs of formal text, using specific examples.
Citations not required.)
\paragraph{Answer}
The LIHEAP is a national program that is in place to assist low income
households that a pay a high proportion of household income to meet their
immediate home energy needs. This program was put into place in 1981 through
a block grant that can be specified as the Omnibus Budget Reconciliation Act of
1981. After some operation time, in 1994 the Congressional Committees realized
that in order to best serve the households with the largest energy burden a
report should be produced. The report that was given to us is from 2005 and uses
data from a 2001 report. The data presented in this and from the more up to date
montana.gov will be used to analyze the efficiency and extent to which this
program is successful in addressing the energy burden concerns of the ACEEE.

The main components/variables addressed by both the ACEEE and the LIHEAP are
extremely similar in identifying energy burden across different demographics.
Both organizations examine energy burden based on the following groups
income level, vulnerability, geographic location, climate, household characteristics,
and housing unit. Along with this both have concluded that energy burden varies
widely among households and that depending on location there are different
expenses related to what the household is spending their income on. Along with
this studies have consistently found that homeowners that have lower income, on
average, have a higher energy burden. The LIHEAP is quite an efficient and effective
program even though roughly only \%20 of eligible households receive assistance.
This statistic is quite disturbing to find and can be attributed to an underfunding
for the grantees that must then make the hard decision on who gets to live in a
habitable house and who must decide between buying food, medicine, or other essentials
and keeping their gas tanks not empty, or heat on to above freezing. Overall,
the LIHEAP is doing the best they can and have identified the critical cases in
a similar manner as the ACEEE but without proper funding their efforts can only
go so far to keeping habitability of homes in a liveable condition.

Through the LIHEAP each state is given the flexibility to create their own system
of distribution of assistance based on the previously defined national standards.
This is essential as their is not a one size fits all to identifying and solving
the energy burden problem. In Montana, there are two programs that are in place
to assist in reducing the energy burden. The first is the implementation of LIHEAP
that is called LIEAP. In Montana this program is a bit better than at a national
level in that if the household qualifies for LIEAP there is an automatic 15\%
discount applied if they are customers of NorthWestern Energy. This is better
than the previously limited option of only assisting those who are accepted but
similarly to at a national level the amount of money in the block grant is a
in dire need for an increase in funding as there are some theories out there
that are assuming that as the program got harder and harder to get accepted into
over the years less and less households decided to apply for this program. This
is unfortunate and I hope that a new proposition will brought to Council in order
to further advance the funding provided to Montana and therefore be available to
bring faith back into the program that has been placed to help those in need.




% %% ... the bibliography
% \newpage
% \bibliographystyle{acm}
% \bibliography{biblio}

\end{document}
