\documentclass[3p,sort]{elsarticle}

\usepackage{lineno,hyperref,url,chemmacros,siunitx}

\modulolinenumbers[5]
\journal{Dr. Julia Haggerty}


%%%%%%%%%%%%%%%%%%%%%%%
%% Elsevier bibliography styles
%%%%%%%%%%%%%%%%%%%%%%%
%% To change the style, put a % in front of the second line of the current style and
%% remove the % from the second line of the style you would like to use.
%%%%%%%%%%%%%%%%%%%%%%%

%% Numbered
\bibliographystyle{model1-num-names}

%% Numbered without titles
%\bibliographystyle{model1a-num-names}

%% Harvard
%\bibliographystyle{model2-names.bst}\biboptions{authoryear}

%% Vancouver numbered
%\usepackage{numcompress}\bibliographystyle{model3-num-names}

%% Vancouver name/year
%\usepackage{numcompress}\bibliographystyle{model4-names}\biboptions{authoryear}

%% APA style
%\bibliographystyle{model5-names}\biboptions{authoryear}

%% AMA style
%\usepackage{numcompress}\bibliographystyle{model6-num-names}

%% `Elsevier LaTeX' style
%\bibliographystyle{elsarticle-num}
%%%%%%%%%%%%%%%%%%%%%%%

\begin{document}

\begin{frontmatter}

\title{Green Hydrogen: The Silver Bullet for Decarbonisation}
\author{Patrick O'Connor}
\address{Geography of Energy Resources, GPHY 326}

%\title{Elsevier \LaTeX\ template\tnoteref{mytitlenote}}
%\tnotetext[mytitlenote]{Fully documented templates are available in the elsarticle package on \href{http://www.ctan.org/tex-archive/macros/latex/contrib/elsarticle}{CTAN}.}

%% Group authors per affiliation:
%\author{Elsevier\fnref{myfootnote}}

%\fntext[myfootnote]{Since 1880.}
%% or include affiliations in footnotes:
%\author[mymainaddress,mysecondaryaddress]{Elsevier Inc}
%\ead[url]{www.elsevier.com}

%\author[mysecondaryaddress]{Global Customer Service\corref{mycorrespondingauthor}}
%\cortext[mycorrespondingauthor]{Corresponding author}
%\ead{support@elsevier.com}

%\address[mymainaddress]{1600 John F Kennedy Boulevard, Philadelphia}
%\address[mysecondaryaddress]{360 Park Avenue South, New York}

\begin{abstract}
World leaders currently face a significant issue in creating energy in a net-zero carbon process. It is 
well known that these carbon emissions are leading the world to an expedited warming that is negatively 
impacting the weather patterns that are experienced by the occupants of the earth. 
This study aims to display two 
of the larger issues surrounding providing energy to an ever-growing population.
    
To examine this issue of providing energy in a net-zero carbon process, extensive research on the current situation and the ideas for moving forward were examined. Through this examination, the silver bullet for decarbonization was discovered. Throughout the second half of the paper, the SGH2 green hydrogen plant is 
examined to determine if this is the silver bullet. 
    
The results of this investigation show that this is a great solution for tackling both energy production and
waste management. 
Although there is some skepticism on whether this project is feasible. After examination these skeptics 
appear all from media sources that can be assumed to be ill-informed or pushing a bias into their writings.
\end{abstract}

\end{frontmatter}

%\linenumbers


%-----------------------------Introduction--------------------------------------
\section{Introduction}
\paragraph{Ad astra per aspera} A Latin phrase that translates to "Through adversity to the stars".
 This classic Latin phrase is literally and philosophically pertinent to solving the renewable 
 energy situation that our world is currently facing. The world as a whole is currently facing 
 hardship in supplying clean energy to the inhabitants living in a variety of areas. Whether 
 this energy is used to power cars, condition their homes, or provide a 
nontoxic cooking method. To combat this issue, we have engineered a variety of solutions from 
biofuels to renewable resources like solar and wind energy. The one that has recently been picking
 up speed as the silver bullet for decarbonization is a combination of the two and also one 
 of the most abundant elements in our Universe, Hydrogen. Being the most prominent element in 
 our Universe at roughly 75\% of our baryonic mass it can be found in places like water, air, 
 and also the stars that surround us.  \cite{forbes}


\paragraph{History of Hydrogen} Considering the abundance in our universe, humans have been 
attempting to harness the energy that is held in compounds of hydrogen since the original 
discovery in 1776. This discovery made by British scientist Henry Cavendish evolved hydrogen 
gas by reacting zinc metal with hydrochloric acid. Later Cavendish applied a spark to hydrogen 
gas which led to the discovery that water \ch{H2O} 
being composed of hydrogen and oxygen. From this many years passed before another scientist 
made the connection that applying electric current to water would produce hydrogen and oxygen 
gases. This process of applying current to \ch{H2O} is termed electrolysis and was a major 
breakthrough in creating and storing
energy. Through expansive investigation, scientists reversed this process to create electrical 
current and water by combining hydrogen and oxygen gases. This discovery
led the scientist, Sir William Grove, to create the first gas battery. With this, many versions 
of fuel cells were created but the first major attempt to utilize this energy was by German Engineer, 
Rudolf Erren, which converted internal combustion engines of trucks, buses, and submarines to use 
hydrogen mixtures. Although successful,
these engines were highly complex but were able to produce an abundance of power. Unfortunately, 
this complexity left hydrogen to be only picked up by sophisticated agencies such as NASA. After 
many years of technological advancement and scientific investigations, many began to see the clean 
energy possibilities such as Iceland, Germany, and lastly the U.S. 
under the Bush administration in 2003. This commit of 1.2 billion dollars of the U.S. led to more 
advancement and later to the creation of the SGH2 green hydrogen plant that could 
be the solution for waste management while also reducing our carbon emissions tenfold. 

\citep{bloomberg}


%--------------------------Concept and Issue------------------------------------
\section{Concepts and Issues}

Before diving into the details of how the SGH2 plant has planned to solve a couple major social 
and environmental problems. 
Laying out some ground level information of the current state of carbon emissions and waste 
management is essential.

\paragraph{Carbon Emissions} The current state of carbon emissions is improving while also not 
being in a great place due to infrastructure that has been progressively compounded into an 
incomprehensible system that cannot change within a short time. Going back to the beginning of 
recorded history on earth, carbon emissions
were not a problem as nature has been set up to work in a symbiotic cycle. Such that plants 
and animals decompose producing \ch{CO2} and this \ch{CO2} is then used by the living plants 
(land \& ocean) through photosynthesis. These plants feed animals that feed larger animals and 
they are killed in one way or another and the cycle repeats. Then the advanced humans entered 
the playground with farming, alternate transportation, and resource collection. These activities 
along with many others break this cycle by releasing extra carbon and other greenhouse gasses 
into the atmosphere. Along with this humans inherently tend to migrate and expand. 
This pattern has led to the cutting down of \ch{CO2} absorbing trees to create new land for 
farming and housing. Overall, humans are counteractive to the natural cycle that has worked in 
the past. This negative impact is important as the extra released \ch{CO2} into the atmosphere 
of the earth traps heat in and absorbs radiation from space. 
As many of us have observed throughout the past years, the excess heat being trapped in our 
atmosphere has led to altered weather patterns including, 
higher average temperature and extreme weather patterns. With this many privileged individuals 
have not noticed too much of a difference besides increased expenses and some irregular seasons. 
Although the privileged have not experienced extreme struggles, those that are of lower-income are 
disproportionately affected by these climate changes and continue to be driven farther into a lower 
quality of life. While \ch{CO2} is not the only greenhouse gas that is of importance without reducing 
the release of \ch{CO2} these households will continue to become more vulnerable to the extremes that 
are ever-increasing daily.

 \citep{carbon}

 
\paragraph{Waste management} The second major social problem, waste management, is closely linked 
to both the carbon cycle including emissions and poverty. Waste management is a major social problem
that has been rising in social awareness. Waste that is not properly being disposed of is currently 
clogging drains, rivers, and oceans. This clogging causes floods and transmits diseases that can 
cause respiratory issues and harming surrounding animals. This poor waste management has led to 
between 400,000 to 1 million people deaths each year in developing countries.  As our communities and 
cities continue to expand in both geographical size and population managing, this death count and tons 
of waste will continue to grow. According to the  World Bank in 2016, globally we produced 2.01 
billion tons of solid waste. With the rapid population growth and urbanization, it is predicted that 
by 2050 the amount of waste will increase by 70\% to 3.40 billion tons in 2050. Just as carbon 
emissions disproportionately affect those of lower income, waste management currently does as well. 
With no waste infrastructure in place, many of these areas or countries are forced to dispose of in 
unregulated dumps or burn openly. This further increases the negative impact that humans have on the 
earth through destroying the complex carbon management system that has evolved over time. 
\citep{waste, tearfund, worldbank}




%------------------------------Case Study---------------------------------------
\section{Case Study}
\paragraph{Lancaster Project} Lancaster, CA is a medium-size town of roughly 58,000 houses north of 
Los Angelos. With an average household income of \$55,000 it is
a lower-middle-class town that suffers from a slightly above average percent of 
impoverished occupants(21.7\%, 2019). 
Although this town is in the lower middle class it has made expansive efforts to 
reduce carbon emissions in the past with initiatives in solar energy production. 
Since the 2008 announcement of a net-zero city campaign all schools have been outfitted 
with a sufficient amount of panels to be net-zero in carbon emission. Along with this
in 2014, a policy was put in place that stated for every house or rooftop built going forward 
at least 2 watts of solar energy must be created per square foot of real estate. With this 
clear desire for their town to produce a low carbon footprint, it is not a surprise that they 
are pursuing new technology to transform their entire energy infrastructure. This driving 
force has led them to find the silver bullet for decarbonization, the SGH2 green hydrogen 
plant. The SGH2 plant was officially signed onto a co-ownership contract in early 2020 and 
has just broken ground in Q1 of this year and should be operational by the fourth quarter of 
2022. Being the largest of its kind, this is an unseen timeline for the largest plant in 
history. It should be able to produce up to 11,000 kg of green hydrogen per day and 3.8 
million kg per year. 
Along with producing an astonishing amount of energy, the Lancaster plant will be able to 
process 42,000 tons of recycled waste annually.

\citep{sgh2, sgh2-ca, sgh2site, solar}

\paragraph{Proprietary Tech} In the past, the extraction of hydrogen from natural resources 
was done through the electrolysis of \ch{H2O} which resulted in a gas form. Although effective 
in producing hydrogen, these systems were, in the end, expensive, used up a lot of energy, and 
were not a zero carbon
output. In SGH2 Lancaster, cost and carbon output is reduced and the abundance of renewable 
energy from Lancaster will be used to run the proprietary 
Solena Plasma Enhanced Gasification (SPEG). This process detailed as the plasma-enhanced thermal 
catalytic conversion is the key to answering all of these previous green hydrogen growing pains. 
This 95\% oxygen-enriched plasma torch runs at roughly 3500 to \SI{4000}{\celsius} and when contact 
is made with solid materials like recycled plastic or metal they disintegrate into their molecule 
compounds without toxic combustion ash. After this these molecules bound into a very high-quality 
hydrogen-rich syngas that is free of tar, soot, and heavy metals. This syngas 
is then sent through a pressure swing absorber system that will result in 99.9999\% pure hydrogen. 

To see a video walk through from SGH2 \href{https://vimeo.com/411145543}{Click Here}.

\citep{sgh2-ca, sgh2site}

\paragraph{Who is involved} This complex system has been developed and tested by many influential 
groups including Berkeley Lab, UC Berkely, Thermosolv, Fluor, 
Integrity Engineers, Millenium, HyetHydrogen, and NASA. With all of these powerful groups, 
many encouraging discoveries have been made that is reassuring for the success of this plant. 
Such as Berkeley Lab's research on carbon dioxide displacement when directly using the 
SGH2 modular system. Through testing displacement in a lab setting, researchers were able to 
discover that on average the SGH2 was above to displace 23 to 31 tons of carbon dioxide for 
every ton of hydrogen produced. This is roughly 13 to 19 tons more \ch{CO2} being avoided per 
ton in the hydrogen production process. Along with this in private studies, it has been found 
that the production cost for a KG of H2 
is roughly 20\% of the cost when comparing the SGH2 plasma torch process(\$2) with Electrolysis 
produced hydrogen(\$10-\$13). Along with it has been proven that this is comparable to the current 
cost of hydrogen produced from fossil fuels which on average cost(\$2-\$6) per kg.

\cite{sgh2site, isegoria}

\paragraph{What this could replace} The current plan for the hydrogen produced at the Lancaster, CA 
plant is to power the transportation sector of the surrounding area. 
California has an existing and increasing hydrogen infrastructure. According to the SGH2 CEO Dr. 
Robert Do, negotiations with the largest 
owners and operators of hydrogen refueling stations in California are happening as you read this. 

Although the above is only pertinent to the Lancaster facility, hydrogen fuel has an ever-growing 
market to take advantage of. Hydrogen has 
the potential to replace natural gas when produced at a low cost and SoCalGas is hoping to take 
advantage of this in the future through the 
Lancaster plant. Along with this many cement plants have the potential to transition their existing 
infrastructure to run on clean energy. 
Lastly in California, various transit agencies in the Los Angelos area have purchased roughly 1000 
new hydrogen buses to create a 100\% 
zero-emissions fleet. Along with this bus line surrounding cities such as San Bernardino has 
announced the commission of their first hydrogen train. See Figure [1] Visual representation of the proximity of major cities below for proximity.

  
At a global level, hydrogen has been noticed by the European Commission as an excellent 
alternative to a variety of sectors and essential to ensuring the European Green Deal and Europe's 
clean energy transition come to fruition. The EU found that the sectors that would be most efficient 
to pursue transitioning are those that are carbon industrial processes, such as steel or chemical 
production. They found through economic studies that the implementation of green hydrogen would lower 
greenhouse gas emissions while also strengthening the global competitiveness for those industries. 
Along with this, these also found the transportation systems to be a prime target for transitioning 
to green hydrogen.
\citep{forbes, isegoria, eu, expert}

\paragraph{Plans for expansion} As previously noted, the SGH2 plants are produced with a modular design and can be built and running within time-spans that have not been 
seen before in hydrogen plants. This design allows complex parts to be assemble off-site reducing time of construction and chance for 
malfunctioning parts. With an expansive amount of supporting research, SGH2 negotiations are currently happening in France, Saudi Arabia, Ukraine,
Greece, Japan, South Korea, Poland, Turkey, Russia, China, Brazil, Malaysia, and Australia.
\cite{sgh2site}

%------------------------------Analysis-----------------------------------------
\section{Analysis}
\paragraph{Destructive Emissions Reduction} The current state of the carbon emissions while improving 
is still on a trajectory to cause major problems in all areas of our lives without improving the 
current energy production. This will be an increasingly prominent question that needs answering as 
the world advances to a more connected ecosystem and countries continue to rise from third and second 
world to first-world countries. With the implementation of 
SGH2 plasma torch technology the carbon footprint for almost all sectors can be reduced to an extremely 
low if not net-zero. I state that it is possible 
that this could be a net-zero system while also 
taking into the fact that there is an extreme cost to transitioning. This transition would be extremely 
resource expensive in that to make a successful switch 
many municipal functions(trains, buses, trucks, etc.) would have to either modify their current vehicles 
of transportation when feasible or make the investment 
to purchase hydrogen-powered vehicles such as California has done. Along with this building one of these 
plants even with a decreased time of construction will 
require a large number of resources to fund the construction and material purchase. Overall, the SGH2 
corporation has successfully found a solution to decreasing 
the harmful carbon emissions produced when creating energy and this should be investigated by countries, 
states, and everyone looking to create massive change 
in the carbon footprint.
\citep{carbon}


\paragraph{Waste management} Along with reducing the carbon footprint, SGH2's ability to use waste to 
create energy is a massive step in the right direction. In the ever-growing world, waste has become 
an uncontrolled process in many places. Trash has already infiltrated nature as a whole and will 
continue to do so if steps are not taken to properly dispose of it. While also looking very bad in 
nature this trash infestation is disrupting what is remaining of the natural carbon cycle ecosystem 
that exists. 
The ocean for example is one of the largest processors for carbon while also assisting in the cooling 
of our atmosphere. I truly believe along with many other scientists that if we are unable to stop the 
destruction of the ocean we will continue to see more extreme weather patterns that will eventually 
lead to the downfall of the earth. Along with this the current situation for those that are in 
lower-income areas trash management and energy production are of serious concern as they are 
currently stuck in a cycle of producing, using, disposing of. This cycle can be streamlined into a 
much more efficient timeline with the ability of 
SGH2's plants to rearrange the cycle by disposing of waste in a manner that is not only creating 
affordable energy but decreasing the waste sent to landfills and beaches thereby being the solution 
to waste management if feasible.
\cite{sgh2site}


\paragraph{Feasibility} Although it appears that this is the silver bullet for decarbonization an 
examination of the feasibility of the transition is essential for considering if this is the path 
to move forward with. This is extremely tricky to do as the SGH2 will be implementing a new 
patented technology that has only been used in labs. 
Although only lab studies can be examined for this technology, there are reassuring facts 
that have been presented by the well-respected Berkeley Labs in California.
Through testing on the modular system, they were able to scientifically prove that the SGH2 
plant will be an affordable net-zero solution to the clean energy initiative. While this is 
reassuring, examining the past attempts of using electrolysis to produce hydrogen from \ch{H2O}. 
The largest downfall that these hydrogen transition projects have faced in the past according to 
the EU is the failure of long-term planning and government-backed investments. There is no doubt 
that these projects are an expensive approach with the SGH2 plant in Lancaster expected to take up 
5 acres of land and cost roughly \$55 million. This large sum of money is unfortunately not 
considering the cost of transitioning the vehicles to run on hydrogen or to change the perspective 
of the surrounding citizens that will have a minor inconvenience when changing their current way of 
life. Although this is initial investment is large, the potential for reducing the cost of processing 
waste is great in that the annual projected landfill expense savings will be roughly \$3 million. 
I have strong faith 
that the annual savins will increase more as these systems become more efficient and that this is a 
feasible option for creating energy and
decreasing carbon emissions while also managing the waste issue that is killing hundreds of 
thousands of humans yearly.

\cite{berkeley,sgh2site, bloomberg}


%------------------------------Summary------------------------------------------
\section{Summary}
\paragraph{The Silver Bullet} All things considered, after researching to find scientific studies, 
media reports, industry, and government documentation the SGH2 plant is the silver bullet for 
decarbonization. While there is an assortment of skepticism around the feasibility of this plant 
coming to fruition. There is more evidence that counteracts this from reliable institutional sources. 
Along with this, the benefits to our current energy production and waste management systems solidify 
that this a worthy path forward even if it is only in the name of science. Through this adversity of 
skeptics, the stars composed of hydrogen will lead us to a world powered by 
net-zero carbon energy while reducing our waste.
  


%----------------------------Bibliography---------------------------------------
\newpage
\bibliography{mybibfile}


\end{document}
